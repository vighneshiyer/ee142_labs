%texexptitled======================================================================
% lab1-gcd
%-----------------------------------------------------------------------
%

\documentclass[11pt]{article}

% Package includes

\usepackage[table]{xcolor}
\usepackage{graphicx}
\usepackage{color}
\usepackage{comment}
\usepackage{multirow}
\usepackage{askmaps}
\usepackage{amssymb}
\usepackage{amsmath}
\usepackage{tikz}
\usepackage{circuitikzgit}
\usetikzlibrary{arrows, positioning, shapes.geometric, circuits.logic.US}
\tikzstyle{line}=[draw]
\tikzstyle{arrow}=[draw, -latex]

% Wrap long URLs with hyphens
\PassOptionsToPackage{hyphens}{url}\usepackage{hyperref}
\usepackage{pdftexcmds}
\usepackage{upquote}
\usepackage{textcomp}
\usepackage{minted}
\usepackage[listings]{tcolorbox}
\usepackage{enumerate}
\usepackage{enumitem}
\usepackage{mathtools}
\DeclarePairedDelimiter{\ceil}{\Big\lceil}{\Big\rceil}

\tcbset{
texexp/.style={colframe=black, colback=lightgray!15,
         coltitle=white,
         fonttitle=\small\sffamily\bfseries, fontupper=\small, fontlower=\small},
     example/.style 2 args={texexp,
title={Question \thetcbcounter: #1},label={#2}},
}

\newtcolorbox{texexp}[1]{texexp}
\newtcolorbox[auto counter]{texexptitled}[3][]{%
example={#2}{#3},#1}

\setlength{\topmargin}{-0.5in}
\setlength{\textheight}{9in}
\setlength{\oddsidemargin}{0in}
\setlength{\evensidemargin}{0in}
\setlength{\textwidth}{6.5in}

% Useful macros

\newcommand{\note}[1]{{\bf [ NOTE: #1 ]}}
\newcommand{\fixme}[1]{{\bf [ FIXME: #1 ]}}
\newcommand{\wunits}[2]{\mbox{#1\,#2}}
\newcommand{\um}{\mbox{$\mu$m}}
\newcommand{\xum}[1]{\wunits{#1}{\um}}
\newcommand{\by}[2]{\mbox{#1$\times$#2}}
\newcommand{\byby}[3]{\mbox{#1$\times$#2$\times$#3}}


\newenvironment{tightlist}
{\begin{itemize}
 \setlength{\parsep}{0pt}
 \setlength{\itemsep}{-2pt}}
{\end{itemize}}

\newenvironment{titledtightlist}[1]
{\noindent
 ~~\textbf{#1}
 \begin{itemize}
 \setlength{\parsep}{0pt}
 \setlength{\itemsep}{-2pt}}
{\end{itemize}}

% Change spacing before and after section headers

\makeatletter
\renewcommand{\section}
{\@startsection {section}{1}{0pt}
 {-2ex}
 {1ex}
 {\bfseries\Large}}
\makeatother

\makeatletter
\renewcommand{\subsection}
{\@startsection {subsection}{1}{0pt}
 {-1ex}
 {0.5ex}
 {\bfseries\normalsize}}
\makeatother

% Reduce likelihood of a single line at the top/bottom of page

\clubpenalty=2000
\widowpenalty=2000

% Other commands and parameters

\pagestyle{myheadings}
\setlength{\parindent}{0in}
\setlength{\parskip}{10pt}

% Commands for register format figures.

\newcommand{\instbit}[1]{\mbox{\scriptsize #1}}
\newcommand{\instbitrange}[2]{\instbit{#1} \hfill \instbit{#2}}

\graphicspath{{./figs/}}


%-----------------------------------------------------------------------
% Document
%-----------------------------------------------------------------------

\begin{document}
\def\PYZsq{\textquotesingle}


\newcommand{\headertext}{EE 142/241A Lab 1 Report Re-do}

\title{\vspace{-0.4in}\Large \bf \headertext \vspace{-0.1in}}
\author{Several Authors}

\date{}
\maketitle

\markboth{\headertext}{\headertext}
\thispagestyle{empty}

\section{Calibration and SMA Dummy Short Modeling}
We perform calibration using the separate male calibration standards (see appendix).
To do this calibration, we need to use a 50ps F-F thru on both cables, we will compensate this out later.
Unfortunately, the F-F thru has some loss that we can't characterize, but it's very small at our band of interest.

To perform the SMA dummy short modeling, we first dial in -50ps delay on the port extension for port 1 and 2. Then S-parameters are captured for modeling.

\textbf{TODO: Insert SMA dummy short modeling results and modeling strategy}

Now the reference plane can be moved to the edge of the SMA dummy short.
To achieve flat phase on $S_{11}$ and $S_{22}$, we need an electrical delay of 76.35 ps, which translates to an SMA dummy short propagation time of 38.175 ps (this closely matches our previous result).

The final port extension value is $-50 + 38.175 = -11.825$ ps.

To achieve flat gain on $S_{11}$ and $S_{22}$, we use these values:

\begin{itemize}
    \item Loss 1: -0.01 dB @ 2.75 Ghz
    \item No Loss 2 or Loss at DC
\end{itemize}

This accomplishes the task of keeping a flat $S_{11}$ curve up until 3 Ghz or so. This also compensates for the loss characteristics of the SMA connector over those of the F-F thru.

\section{Thru Board T-Line}

\textbf{TODO: thru\_refedges.s2p data and plots}

This doesn't seem to work properly as I only need around 80ps electrical delay to flatten the $S_{21}$ phase response. Clearly something is getting counted twice here, but I can't find out what.

\section{Open in T-Line Model}

\section{Via Model - ShuntShort}

\section{0603 Package Parasitics Model w/ $0 \Omega$ Resistor}

\section{0603 Inductor Parasitics Model}

\section{0603 Capacitor Parasitics Model}

\section{0603 Capacitor HF Analysis Model}

\appendix
\section{Calibration Details on the E5071C VNA}
The E-cal and the female/male calibration triangles are broken or don't provide good enough calibrations.
The failure mode seen with these calibration standards was occasional gain (more than 0dB return loss) on $S_{11}$ at a particular band when a short or open was connected after calibration.

We also noticed that the female separated calibration standards didn't give a very good result. The only standards that work well are the male separated standards.

When using the male cal standards, (85052-60006 - Short, 85052-60008 - Open, 902C - Load), a F-F thru needs to be used with a known delay of 51 ps, and sometimes a M-M thru needs to be used with a known delay of 98.9 ps.
When performing transmission calibration (2-port), use the M-M thru and the delay value of 98.9 ps.
This M-M thru described above is the SMA thru, \emph{NOT} the 3.5mm thru. You will know which one is which when inspecting the dielectric.

\textbf{Conclusion: Use separated male standards from 85052E.}
\section{Notes from Previous Lab 1 Attempt}

Port extensions performed on port 1 and 2 to move reference plane to end of SMA connector:
\begin{itemize}
    \item 38 ps delay (13.7mm equivalent electrical length on port 1 and 13.67mm on port 2)
    \item Loss @ DC = 0 dB
    \item Loss 1: 33mdB @ 1 Ghz
    \item Loss 2: -23mdB @ 500 Mhz
\end{itemize}

Attenuation per frequency measurements taken with the SMA dummy short:
\begin{itemize}
    \item Port 1 (598.19 Mhz, -0.04771 dB), (1.352510 Ghz, -0.08127 dB)
    \item Port 2 (598.19 Mhz, -0.05873 dB), (1.352510 Ghz, -0.08767 dB)
\end{itemize}

Measurements performed with the thru board:
\begin{itemize}
    \item Length of thru board $d_0 = 0.926 \text{ in} = 2.352 \text{ cm}$
    \item 156 ps of electrical delay $\tau_0$ required to produce flat phase on $S_{21}$
    \item $v = c_p = \frac{d_0}{\tau_0} = 1.507e8 \text{ m/s}$
    \item $\epsilon_{eff} = 3.96$ (compared to FR4 typical 4.4)
    \item $Z_{0,measured} = 49.2 \Omega$ at 4.22 Ghz
\end{itemize}

Calibration to get a 0.7mm length transmission line:
\begin{itemize}
    \item Port extend port 1 and 2 with 109ps to form a 0.7mm thru line on the thru board.
    \item To make the loss maximally flat at 0dB around 600 Mhz use:
    \begin{itemize}
        \item Loss 1: 74 mdB @ 1 Ghz
        \item Loss 2: -35 mdB @ 500 Mhz
    \end{itemize}
\end{itemize}

We measured 0.44 in from the left of the open board and 0.462 in from the right of the open board to the gap. The distance of the gap was 0.033 in. We then calculated a delay of 112.16 ps from the left and 115.869 ps from the right to move the reference plane to the edge of the open.

For the rest of this lab we used a 8.7 nH inductor and 8 pF capacitor.

The rest of this lab can be performed with these measurements alone.
\end{document}
